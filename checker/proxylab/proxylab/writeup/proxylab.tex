\documentclass[11pt]{article}
\usepackage{times}

%% Page layout
\oddsidemargin 0pt
\evensidemargin 0pt
\topmargin -48pt
\textheight 648pt
\textwidth 469pt
\setlength{\parindent}{0em}
\setlength{\parskip}{1ex}


\title{15-213, Fall 200X\\
Lab Assignment 7: Web Proxy\\
Assigned: November XXX, Due: December XXX, 11:59PM}

\author{}
\date{}

\begin{document}

\maketitle

Harry Bovik ({\tt bovik@cs.cmu.edu}) is the lead person for this
assignment.

\section*{Introduction}

A Web proxy is a program that acts as a middleman between a Web
browser and an {\em end server}. Instead of contacting the end server
directly to get a Web page, the browser contacts the proxy, which
forwards the request on to the end server. When the end server replies to the
proxy, the proxy sends the reply on to the browser.

Proxies are used for many purposes. Sometimes proxies are used in
firewalls, such that the proxy is the only way for a browser inside
the firewall to contact an end server outside. The proxy may do translation
on the page, for instance, to make it viewable on a Web-enabled cell
phone.  Proxies are also used as {\em anonymizers}. By stripping a request
of all identifying information, a proxy can make the browser anonymous
to the end server.  Proxies can even be used to cache Web objects, by
storing a copy of, say, an image when a request for it is first made,
and then serving that image in response to future requests rather than
going to the end server.

In this lab, you will write a concurrent Web proxy that logs requests.
In the first part of the lab, you will write a simple sequential proxy
that repeatedly waits for a request, forwards the request to the end
server, and returns the result back to the browser, keeping a log of
such requests in a disk file. This part will help you understand
basics about network programming and the HTTP protocol.

In the second part of the lab, you will upgrade your proxy so that it
uses threads to deal with multiple clients concurrently. This part
will give you some experience with concurrency and synchronization,
which are crucial computer systems concepts.

\section*{Logistics}

As always, you may work in a group of up to two people.  The only
handin will be electronic.  Any clarifications and revisions to the
assignment will be posted on the course Web page.

\section*{Hand Out Instructions}

\begin{quote}
\bf SITE-SPECIFIC: Insert a paragraph here that explains how the instructor
will hand out the \texttt{proxylab-handout.tar} file to the students. 
\end{quote}

Start by copying \texttt{proxylab-handout.tar} to a (protected) directory
in which you plan to do your work.  Then give the command
``\verb@tar xvf proxylab-handout.tar@''.  This will cause a number of files
to be unpacked in the directory:

\begin{itemize}
\item \texttt{proxy.c}: This is the only file you will be 
modifying and handing in. It contains the bulk of the logic
for your proxy.
\item \texttt{csapp.c}: This is the file of the same name that
is described in the CS:APP textbook. It contains error handling
wrappers and helper functions such as the RIO (Robust I/O) package
(CS:APP 11.4), \texttt{open\_clientfd} (CS:APP 12.4.4), and {\tt
open\_listenfd} (CS:APP 12.4.7).

\item \texttt{csapp.h}:  This file contains a few manifest constants, type 
definitions, and prototypes for the functions in \texttt{csapp.c}.

\item \texttt{Makefile}: Compiles and links \texttt{proxy.c} and 
\texttt{csapp.c} into the executable \texttt{proxy}.

\end{itemize}

Your \texttt{proxy.c} file may call any function in the
\texttt{csapp.c} file. However, since you are only handing in 
a single \texttt{proxy.c} file, please don't modify 
the \texttt{csapp.c} file. If you want different versions of 
functions in in \texttt{csapp.c} (see the Hints section), 
write new functions in the \texttt{proxy.c} file.


\section*{Part I: Implementing a Sequential Web Proxy}

In this part you will implement a sequential logging proxy.  Your
proxy should open a socket and listen for a connection request. When
it receives a connection request, it should accept the connection,
read the HTTP request, and parse it to determine the name of the end
server.  It should then open a connection to the end server, send it
the request, receive the reply, and forward the reply to the browser
if the request is not blocked.

Since your proxy is a middleman between client and end server, it will
have elements of both. It will act as a server to the web browser, and
as a client to the end server. Thus you will get experience with both
client and server programming.

\subsection*{Logging}

Your proxy should keep track of all requests in a log file named {\tt
proxy.log}. Each log file entry should be a file of the form:

\begin{verbatim}
Date: browserIP URL size
\end{verbatim}

where \texttt{browserIP} is the IP address of the browser, \texttt{URL} is the
URL asked for, \texttt{size} is the size in bytes of the object that was
returned.  For instance:

\begin{verbatim}
Sun 27 Oct 2002 02:51:02 EST: 128.2.111.38 http://www.cs.cmu.edu/ 34314
\end{verbatim}

Note that \texttt{size} is essentially the number of bytes received
from the end server, from the time the connection is opened to the time it
is closed. Only requests that are met by a response from an end server
should be logged.  We have provided the function \texttt{format\_log\_entry} 
in \texttt{csapp.c} to create a log entry in the required format.

\subsection*{Port Numbers}

You proxy should listen for its connection requests on the port number
passed in on the command line:
\begin{verbatim}
unix> ./proxy 15213
\end{verbatim}
You may use any port number $p$, where $1024 \le  p \le 65536$, and
where $p$ is not currently being used by any other system or user services
(including other students' proxies). See \texttt{/etc/services} for a
list of the port numbers reserved by other system services.

\section*{Part II: Dealing with multiple requests concurrently}

Real proxies do not process requests sequentially. They deal with
multiple requests concurrently. Once you have a working sequential
logging proxy, you should alter it to handle multiple requests
concurrently.  The simplest approach is to create a new thread to deal
with each new connection request that arrives (CSAPP 13.3.8).

With this approach, it is possible for multiple peer threads to access
the log file concurrently. Thus, you will need to use a semaphore to
synchronize access to the file such that only one peer thread can
modify it at a time.  If you do not synchronize the threads, the log
file might be corrupted.  For instance, one line in the file might
begin in the middle of another.

\section*{Evaluation}

Each group will be evaluated on the basis of a demo to your
instructors. See the course Web page for instructions on how to sign
up for your demos.

\begin{itemize}
\item Basic proxy functionality (30 points).  
Your sequential proxy should correctly accept connections, forward the
requests to the end server, and pass the response back to the browser,
making a log entry for each request. Your program should be able to
proxy browser requests to the following Web sites and correctly log 
the requests:
\begin{itemize}
\item \texttt{http://www.yahoo.com}
\item \texttt{http://www.aol.com}
\item \texttt{http://www.nfl.com}
\end{itemize}

\item Handling concurrent requests (20 points). 

Your proxy should be able to handle multiple concurrent connections.
We will determine this using the following test: (1) Open a connection
to your proxy using \texttt{telnet}, and then leave it open without
typing in any data.  (2) Use a Web browser (pointed at your proxy) to
request content from some end server.

Furthermore, your proxy should be thread-safe, protecting all updates
of the log file and protecting calls to any thread unsafe functions
such as \texttt{gethostbyaddr}.  We will determine this by inspection
during the demo.

\item Style (10 points).  
Up to 10 points will be awarded for code that is readable and well
commented. Your code should begin with a comment block that describes
in a general way how your proxy works. Furthermore, each function
should have a comment block describing what that function does.
Furthermore, your threads should run detached, and your code should
not have any memory leaks.  We will determine this by inspection
during the demo.

\end{itemize}



\section*{Hints}

\begin{itemize}
\item The best way to get going on your proxy is to start with the
basic echo server (CS:APP 12.4.9) and then gradually add
functionality that turns the server into a proxy.

\item Initially, you should debug your proxy using telnet as the client (CS:APP
12.5.3). 

\item Later, test your proxy with a real browser. Explore the browser
settings until you find ``proxies'', then enter the host and port
where you're running yours. With Netscape, choose Edit, then
Preferences, then Advanced, then Proxies, then Manual Proxy
Configuration. In Internet Explorer, choose Tools, then Options,
then Connections, then LAN Settings. Check 'Use proxy server,' and
click Advanced. Just set your HTTP proxy, because that's all your
code is going to be able to handle.

\item Since we want you to focus on network programming issues for this
lab, we have provided you with two additional helper routines:
\texttt{parse\_uri}, which extracts the hostname, path, and port components from
a URI, and \texttt{format\_log\_entry}, which constructs an entry for
the log file in the proper format.

\item Be careful about memory leaks. When the processing for an HTTP
request fails for any reason, the thread must close all open socket
descriptors and free all memory resources before terminating.

\item You will find it very useful to assign each thread a small unique
integer ID (such as the current request number) and then pass this
ID as one of the arguments to the thread routine. If you display
this ID in each of your debugging output statements, then you can
accurately track the activity of each thread.

\item To avoid a potentially fatal memory leak, your threads should run
as detached, not joinable (CS:APP 13.3.6).

\item Since the log file is being written to by multiple threads, you
must protect it with mutual exclusion semaphores whenever you write
to it (CS:APP 13.5.2 and 13.5.3).

\item Be very careful about calling thread-unsafe functions such as
\texttt{inet\_ntoa}, \texttt{gethostbyname}, and \texttt{gethostbyaddr}
inside a thread.  In
particular, the \texttt{open\_clientfd} function in \texttt{csapp.c} 
is thread-unsafe because it calls \texttt{gethostbyaddr}, a Class-3 thread
unsafe function (CSAPP 13.7.1). You will need to write a
thread-safe version of \texttt{open\_clientfd}, called \texttt{open\_clientfd\_ts}, 
that uses the lock-and-copy technique (CS:APP 13.7.1) when it calls
\texttt{gethostbyaddr}.

\item Use the RIO (Robust I/O) package (CS:APP 11.4) for all 
I/O on sockets. Do not use standard I/O on sockets. You will quickly
run into problems if you do. However, standard I/O calls such as
\texttt{fopen} and \texttt{fwrite} are fine for I/O on the log file.

\item The \texttt{Rio\_readn}, \texttt{Rio\_readlineb}, and \texttt{Rio\_writen}
error checking wrappers in \texttt{csapp.c} are not appropriate for a
realistic proxy because they terminate the process when they
encounter an error. Instead, you should write new wrappers called
\texttt{Rio\_readn\_w}, \texttt{Rio\_readlineb\_w}, and
\texttt{Rio\_writen\_w} that simply return after printing a warning
message when I/O fails. When either of the read wrappers detects an
error, it should return 0, as though it encountered EOF on the socket.

\item Reads and writes can fail for a variety of reasons. The most
common read failure is an \texttt{errno = ECONNRESET} error caused by
reading from a connection that has already been closed by the peer on
the other end, typically an overloaded end server.
The most common write failure is
an \texttt{errno = EPIPE} error caused by writing to a connection that
has been closed by its peer on the other end. This can occur for
example, when a user hits their browser's Stop button during a long
transfer.

\item Writing to connection that has been closed by the peer first time
elicits an error with errno set to EPIPE. Writing to such a
connection a second time elicits a SIGPIPE signal whose default
action is to terminate the process. To keep your proxy from
crashing you can use the SIG\_IGN argument to the signal function
(CS:APP 8.5.3) to explicitly ignore these SIGPIPE signals

\end{itemize}

\section*{Handin Instructions}

\begin{quote}
\bf SITE-SPECIFIC: Insert a paragraph here that tells each team how to
hand in their \texttt{proxy.c} solution file. For example, here are 
the handin instructions we use at CMU.
\end{quote}

\begin{itemize}
\item Remove any extraneous print statements.
\item Make sure that you have included your identifying information in
  \texttt{proxy.c}.  
\item Create a team name of the form:
  \begin{itemize}
  \item ``ID'' where ID is your andrew ID.
  \end{itemize}
\item To hand in your \texttt{proxy.c} file, type:

\begin{verbatim}
make handin TEAM=teamname
\end{verbatim}

  where \texttt{teamname} is the team name described above.

\item After the handin, you can submit a revised copy by typing

\begin{verbatim}
make handin TEAM=teamname VERSION=2
\end{verbatim}

You can verify your handin by looking at

\begin{verbatim}
/afs/cs.cmu.edu/academic/class/15213-f02/L7/handin
\end{verbatim}
You have list and insert permissions in this directory, but no
read or write permissions.

\end{itemize}

\end{document}
