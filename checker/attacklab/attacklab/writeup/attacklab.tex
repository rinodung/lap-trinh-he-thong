\documentclass[11pt]{article}

\usepackage{times}
\usepackage{alltt}
\usepackage{graphicx}
\usepackage{url}

\bibliographystyle{plain}

\newcommand{\reg}[1]{\textrm{\texttt{\%#1}}}
\newcommand{\xmmreg}[1]{\reg{xmm{}#1}}
\newcommand{\ymmreg}[1]{\reg{ymm{}#1}}

\newcommand{\eaxreg}{\reg{eax}}
\newcommand{\ebxreg}{\reg{ebx}}
\newcommand{\ecxreg}{\reg{ecx}}
\newcommand{\edxreg}{\reg{edx}}
\newcommand{\esireg}{\reg{esi}}
\newcommand{\edireg}{\reg{edi}}
\newcommand{\espreg}{\reg{esp}}
\newcommand{\ebpreg}{\reg{ebp}}
\newcommand{\eipreg}{\reg{eip}}

\newcommand{\raxreg}{\reg{rax}}
\newcommand{\rbxreg}{\reg{rbx}}
\newcommand{\rcxreg}{\reg{rcx}}
\newcommand{\rdxreg}{\reg{rdx}}
\newcommand{\rsireg}{\reg{rsi}}
\newcommand{\rdireg}{\reg{rdi}}
\newcommand{\rspreg}{\reg{rsp}}
\newcommand{\rbpreg}{\reg{rbp}}


%% Page layout
\oddsidemargin 0pt
\evensidemargin 0pt
\textheight 600pt
\textwidth 469pt
\setlength{\parindent}{0em}
\setlength{\parskip}{1ex}

%% ccode environment- for displaying formatted C code (c2tex) 
\newenvironment{ccode}%
{\small}%
{}

%% tty - for displaying TTY input and output
\newenvironment{tty}%
{\small\begin{alltt}}%
{\end{alltt}}




\begin{document}

\title{15-213, Fall 20xx\\
The Attack Lab: Understanding Buffer Overflow Bugs\\
Assigned: Tue, Sept.~29\\
{\bf Due: Thu, Oct.~8, 11:59PM EDT}\\
Last Possible Time to Turn in: Sun, Oct.~11, 11:59PM EDT
}

\author{}
\date{}

\maketitle

\section{Introduction}

This assignment involves generating a total of five attacks on two
programs having different security vulnerabilities.
Outcomes you will gain from this lab include:
\begin{itemize}
\item You will learn different ways that attackers can exploit
  security vulnerabilities when programs do not safeguard themselves
  well enough against buffer overflows.
\item Through this, you will get a better understanding of how to write
  programs that are more secure, as well as some of the features
  provided by compilers and operating systems to make programs less
  vulnerable.
\item You will gain a deeper understanding of the stack and
  parameter-passing mechanisms of x86-64 machine code.
\item You will gain a deeper understanding of how x86-64 instructions
  are encoded.
\item You will gain more experience with debugging tools such as \textsc{gdb}
  and \textsc{objdump}.
\end{itemize}

{\bf Note:} In this lab, you will gain firsthand experience with
methods used to exploit security weaknesses in operating systems and
network servers.  Our purpose is to help you learn about the runtime
operation of programs and to understand the nature of these security
weaknesses so that you can avoid them when you write system code.  We do
not condone the use of any other form of attack to gain
unauthorized access to any system resources.

You will want to study Sections 3.10.3 and 3.10.4 of the CS:APP3e book
as reference material for this lab.

\section{Logistics}

As usual, this is an individual project.  You will generate attacks
for target programs that are custom generated for you.  

\subsection{Getting Files}

You can obtain your files by pointing your Web browser at:

\begin{verbatim}
    http://$Attacklab::SERVER_NAME:15513/
\end{verbatim}

\begin{verbatim}
INSTRUCTOR: $Attacklab::SERVER_NAME is the machine that runs the
 attacklab servers. You define it in attacklab/Attacklab.pm and in
 attacklab/src/build/driverhdrs.h
\end{verbatim}

The server will build your files and return them to your browser in a
\texttt{tar} file called \texttt{target}$k$\texttt{.tar}, where $k$ is the unique
number of your target programs.

{\bf Note:} It takes a few seconds to build and download your target, so please be patient. 

Save the \texttt{target}$k$\texttt{.tar} file in a (protected) Linux
directory in which you plan to do your work.  Then give the command:
\texttt{tar~-xvf target}$k$\texttt{.tar}.  This will extract a
directory {\tt target}$k$ containing the files described below.

You should only download one set of files. If for some reason you download multiple
targets, choose one target to work on and delete the rest.

{\bf Warning:} If you expand your \texttt{target}$k$\texttt{.tar} on a
PC, by using a utility such as Winzip, or letting your browser do the
extraction, you'll risk resetting permission bits on the executable
files.  

The files in {\tt target}$k$ include:
\begin{description}
\item{\tt README.txt}: A file describing the contents of the directory
\item{\tt ctarget}: An executable program vulnerable to {\em code-injection}
  attacks
\item{\tt rtarget}: An executable program vulnerable to {\em
  return-oriented-programming} attacks
\item{\tt cookie.txt}: An 8-digit hex code that you will use as a
  unique identifier in your attacks.
\item {\tt farm.c}: The source code of your target's ``gadget farm,''
which you will use in generating return-oriented programming attacks.
\item {\tt hex2raw}: A utility to generate attack strings.
\end{description}

In the following instructions, we will assume that you have copied the
files to a protected local directory, and that you are
executing the programs in that local directory.

\subsection{Important Points}

Here is a summary of some important rules regarding valid solutions
for this lab.  These points will not make much sense when you read
this document for the first time.  They are presented here as a
central reference of rules once you get started.

\begin{itemize}
\item
You must do the assignment on a machine that is similar to the one that generated 
your targets. 

\item Your solutions may not use attacks to circumvent the validation
  code in the programs.  Specifically, any address you incorporate
  into an attack string for use by a {\tt ret} instruction should be
  to one of the following destinations:
\begin{itemize}
\item The addresses for functions {\tt touch1}, {\tt touch2}, or {\tt touch3}.

\item The address of your injected code

\item The address of one of your gadgets from the gadget farm.
\end{itemize}

\item You may only construct gadgets from file {\tt rtarget} with
  addresses ranging between those for functions \verb@start_farm@ and
  \verb@end_farm@.

\end{itemize}

\section{Target Programs}

Both {\sc ctarget} and {\sc rtarget}
read strings from standard input. They do so
with the function {\tt getbuf} defined below:

\begin{ccode}
\begin{alltt}
{\scriptsize   1} unsigned getbuf()
{\scriptsize   2} \verb:{:
{\scriptsize   3}     char buf[BUFFER_SIZE];
{\scriptsize   4}     Gets(buf);
{\scriptsize   5}     return 1;
{\scriptsize   6} \verb:}:
\end{alltt}

\end{ccode}

The function {\tt Gets} is similar to the standard library function
{\tt gets}---it reads a string from standard input (terminated by
`\verb@\n@' or end-of-file) and stores it (along with a null
terminator) at the specified destination.  In this code, you can see
that the destination is an array {\tt buf}, declared as having
\verb@BUFFER_SIZE@ bytes.  At the time your targets were
generated, \verb@BUFFER_SIZE@ was a compile-time constant
specific to your version of the programs.

Functions {\tt Gets()} and {\tt gets()} have no way to determine
whether their destination buffers
are large enough to store the string they read.  They simply copy
sequences of bytes,
possibly overrunning the bounds of the storage
allocated at the destinations.

If the string typed by the user and read by {\tt getbuf} is
sufficiently short, 
it is clear that {\tt getbuf} will return 1, as shown
by the following execution examples:

\begin{tty}
    unix>{\em ./ctarget}
    Cookie: 0x1a7dd803
    Type string:{\em Keep it short!}
    No exploit.  Getbuf returned 0x1
    Normal return
\end{tty}

Typically an error occurs if you type a long string:

\begin{tty}
    unix>{\em ./ctarget}
    Cookie: 0x1a7dd803
    Type string:{\em This is not a very interesting string, but it has the property ...}
    Ouch!: You caused a segmentation fault!
    Better luck next time
\end{tty}

(Note that
the value of the cookie shown will differ from yours.)
Program \textsc{rtarget} will have the same behavior.  
As the error message indicates, overrunning the buffer typically
causes the program state to be corrupted, leading to a memory access
error.  Your task is to be more clever with the strings you feed {\sc
ctarget} and {\sc rtarget} so that they do more interesting things.  These are called
{\em exploit} strings.

Both {\sc ctarget} and {\sc rtarget} take several different command line arguments:
\begin{description}
\item[{\tt -h}:] Print list of possible command line arguments
\item[{\tt -q}:] Don't send results to the grading server
\item[{\tt -i FILE}:] Supply input from a file, rather than from
  standard input
\end{description}

Your exploit strings will typically contain byte values that do not
correspond to the ASCII values for printing characters.  The program
{\sc hex2raw} will enable you to generate these {\em raw} strings.  See
Appendix \ref{app:hex2raw} for more information on how to use {\sc hex2raw}.

\textbf{Important points:}

\begin{itemize}

\item Your exploit string must not contain byte value {\tt 0x0a} at
any intermediate position, since this is the ASCII code for newline
(`\verb@\n@').  When {\tt Gets} encounters this byte, it will assume
you intended to terminate the string.

\item {\sc hex2raw} expects two-digit hex values separated by one or
  more white spaces.
So if you want to create a byte with a hex value of 0, 
you need to write it as {\tt 00}.  To create the word {\tt 0xdeadbeef} you 
should pass ``\texttt{ef be ad de}'' to  {\sc hex2raw} (note the reversal
required for little-endian byte ordering).

\end{itemize}

When you have correctly solved one of the levels, your target program
will automatically send a notification to the grading server.  For example:
\begin{tty}
    unix>{\em ./hex2raw < ctarget.l2.txt | ./ctarget}
    Cookie: 0x1a7dd803
    Type string:Touch2!: You called touch2(0x1a7dd803)
    Valid solution for level 2 with target ctarget
    PASSED: Sent exploit string to server to be validated.
    NICE JOB!
\end{tty}

The server will test your exploit string to make sure it really works,
and it will update the Attacklab scoreboard page indicating that your
userid (listed by your target number for anonymity) has completed this phase.

You can view the scoreboard by pointing your Web browser at 
\begin{verbatim}
    http://$Attacklab::SERVER_NAME:15513/scoreboard
\end{verbatim}

Unlike the Bomb Lab, there is no penalty for making mistakes in this
lab.  Feel free to fire away at {\sc ctarget} and {\sc rtarget} with
any strings you like.

IMPORTANT NOTE: You can work on your solution on any Linux machine, but 
in order to submit your solution, you will need to be running on one of the 
following machines:
\begin{verbatim}
INSTRUCTOR: Insert the list of the legal domain names that you
 established in buflab/src/config.c.
\end{verbatim}


\begin{figure}
\begin{center}
\begin{tabular}{|c|c|c|c|c|c|}
\hline
Phase & Program & Level & Method & Function & Points \\
\hline
1 & {\sc ctarget} & 1 & CI & \texttt{touch1} & 10 \\
2 & {\sc ctarget} & 2 & CI & \texttt{touch2} & 25 \\
3 & {\sc ctarget} & 3 & CI & \texttt{touch3} & 25 \\
\hline
4 & {\sc rtarget} & 2 & ROP & \texttt{touch2} & 35 \\
5 & {\sc rtarget} & 3 & ROP & \texttt{touch3} & 5 \\
\hline
\multicolumn{1}{l}{CI:} & \multicolumn{4}{l}{Code injection} \\
\multicolumn{1}{l}{ROP:} & \multicolumn{4}{l}{Return-oriented programming} \\
\end{tabular}
\end{center}
\caption{Summary of attack lab phases}
\label{fig:phases}
\end{figure}
Figure \ref{fig:phases} summarizes the five phases of the lab.  As can
be seen, the first three involve code-injection (CI) attacks on {\sc
  ctarget}, while the last two involve return-oriented-programming
(ROP) attacks on {\sc rtarget}.

\section{Part I: Code Injection Attacks}

For the first three phases, your exploit strings will attack {\sc
  ctarget}.  This program is set up in a way that the stack positions
will be consistent from one run to the next and so that data on the
stack can be treated as executable code.  These features make the program
vulnerable to attacks where the exploit strings contain the byte
encodings of executable code.

\subsection{Level 1}

For Phase 1, you will not inject new code.
Instead, your exploit string will redirect the program to execute 
an existing procedure.

Function {\tt getbuf} is called within {\sc ctarget} by a function
{\tt test} having the following C code:

\begin{ccode}
\input test-c
\end{ccode}

When {\tt getbuf} executes its return statement (line 5 of {\tt
getbuf}), the program ordinarily resumes execution within function
{\tt test} (at line 5 of this function).
We want to change this behavior.
Within the file {\tt ctarget}, there is code for a function {\tt touch1} having
the following C representation:

\begin{ccode}
\input touch1-c
\end{ccode}

Your task is to get {\sc ctarget} to execute the code for {\tt touch1}
when {\tt getbuf} executes its return statement, rather than returning
to {\tt test}.  Note that your exploit string may also corrupt parts
of the stack not directly related to this stage, but this will not
cause a problem, since {\tt touch1} causes the program to exit
directly.

{\bf Some Advice}:
\begin{itemize}

\item All the information you need to devise your exploit string for
this level can be determined by examining a disassembled version of
{\sc ctarget}. Use {\tt objdump -d} to get this dissembled version.

\item
The idea is to position a byte representation of the starting address
for {\tt touch1} so that the {\tt ret} instruction at the end of the
code for {\tt getbuf} will transfer control to {\tt touch1}.

\item
Be careful about byte ordering.  

\item
You might want to use {\sc gdb} to step the program through the last
few instructions of {\tt getbuf} to make sure it is doing the right
thing.

\item
The placement of {\tt buf} within the stack frame for {\tt getbuf}
depends on the value of compile-time constant \verb@BUFFER_SIZE@, as
well the allocation strategy used by {\sc gcc}.
You will need to examine the disassembled code to determine its position.
\end{itemize}


\subsection{Level 2}

Phase 2 involves injecting a small amount of code as part of your exploit string.

Within the file {\tt ctarget} there is code for a function {\tt touch2}
having the following C representation:

\begin{ccode}
\input touch2-c
\end{ccode}

Your task is to get {\sc ctarget} to execute the
code for {\tt touch2} rather than returning to {\tt test}.  In this
case, however, you must make it appear to {\tt touch2} as if you have
passed your cookie as its argument.

{\bf Some Advice}:
\begin{itemize}

\item You will want to position a byte representation of the address
  of your injected code in such a way that {\tt ret} instruction at
  the end of the code for \texttt{getbuf} will transfer control to it.

\item Recall that the first argument to a function is passed in
  register \rdireg{}.

\item Your injected code should set the register to your cookie, and
  then use a \texttt{ret} instruction to transfer control to the first
  instruction in \texttt{touch2}.

\item Do not attempt to use {\tt jmp} or {\tt call} instructions in
  your exploit code.  The encodings of destination addresses for these
  instructions are difficult to formulate.  Use {\tt ret} instructions for
  all transfers of control, even when you are not returning from a
  call.

\item See the discussion in Appendix \ref{app:bytecode}
on how to use tools to generate the byte-level representations of instruction sequences.
\end{itemize}

\subsection{Level 3}

Phase 3 also involves a code injection attack, but passing a string as argument.

Within the file {\tt ctarget} there is code for functions {\tt
  hexmatch} and {\tt touch3}
having the following C representations:

\begin{ccode}
\input touch3-c
\end{ccode}

Your task is to get {\sc ctarget} to execute the
code for {\tt touch3} rather than returning to {\tt test}.
You must make it appear to {\tt touch3} as if you have
passed a string representation of your cookie as its argument.

{\bf Some Advice}:
\begin{itemize}

\item You will need to include a string representation of your cookie
  in your exploit string.  The string should consist of the eight hexadecimal
  digits (ordered from most to least significant) without a leading ``\texttt{0x}.''

\item Recall that a string is represented in C as a sequence of bytes
  followed by a byte with value 0.  Type ``\texttt{man ascii}'' on any Linux
  machine to see the byte representations of the characters you need.

\item Your injected code should set register
\rdireg{}
to the address of this string.

\item When functions {\tt hexmatch} and {\tt strncmp} are called, they
  push data onto the stack, overwriting portions of memory that held
  the buffer used by {\tt getbuf}.  As a result, you will need to be careful where
  you place the string representation of your cookie.
\end{itemize}



\section{Part II: Return-Oriented Programming}

Performing code-injection attacks on program {\sc rtarget} is much
more difficult than it is for {\sc ctarget}, because it uses
two techniques to thwart such attacks:
\begin{itemize}
\item It uses randomization so that the stack positions differ from
  one run to another.  This makes it impossible to determine where your
  injected code will be located.
\item It marks the section of memory holding the stack as
  nonexecutable, so even if you could set the program counter to the
  start of your injected code, the program would fail with a
  segmentation fault.
\end{itemize}

\begin{figure}
\centerline{\includegraphics*[scale=0.8]{rop}}
\caption{Setting up sequence of gadgets for execution.  Byte
value \texttt{0xc3} encodes the \texttt{ret} instruction.}
 \label{fig:rop}
\end{figure}

Fortunately, clever people have devised strategies for getting useful
things done in a program by executing existing code, rather than
injecting new code.  The most general form of this is referred to as
{\em return-oriented programming} (ROP)
\cite{roemer-2012,schwartz-2011}.  The strategy with ROP is to
identify byte sequences within an existing program that consist of one
or more instructions followed by the instruction \texttt{ret}.  Such
a segment is referred to as a {\em gadget}.  Figure \ref{fig:rop}
illustrates how the stack can be set up to execute a sequence of $n$
gadgets.  In this figure, the stack contains a sequence of gadget
addresses.  Each gadget consists of a series of instruction
bytes, with the final one being \texttt{0xc3}, encoding the
\texttt{ret} instruction.
When the program executes a {\tt ret} instruction starting with this
configuration, it will initiate a chain of gadget executions,
with the {\tt ret} instruction at the end of each gadget causing the
program to jump to the beginning of the next.

A gadget can make use of code corresponding to assembly-language
statements generated by the compiler, especially ones at the ends of 
functions.  In practice, there may be some useful gadgets of this form,
but not enough to implement many important operations.  For example,
it is highly unlikely that a compiled function would have {\tt popq
  \rdireg} as its last instruction before {\tt ret}.  Fortunately,
with a byte-oriented instruction set, such as x86-64, a gadget can
often be found by extracting patterns from other parts of the
instruction byte sequence.

For example, one version of {\tt rtarget} contains code generated for the 
following C function:

\begin{ccode}
\begin{alltt}
void setval_210(unsigned *p)
\verb:{:
    *p = 3347663060U;
\verb:}:
\end{alltt}

\end{ccode}

The chances of this function being useful for attacking a system seem
pretty slim.  But, the disassembled 
machine code for this function shows an interesting byte sequence:

\begin{ccode}
\begin{alltt}
0000000000400f15 <setval_210>:
  400f15:       c7 07 d4 48 89 c7       movl   $0xc78948d4,(%rdi)
  400f1b:       c3                      retq   
\end{alltt}

\end{ccode}

The byte sequence {\tt 48 89 c7} encodes the instruction
{\tt movq \raxreg, \rdireg}.
(See Figure \ref{fig:encode}A for the encodings of useful {\tt movq} instructions.)
This sequence is followed by byte value {\tt c3}, which encodes the
{\tt ret} instruction.
The function starts at address {\tt 0x400f15}, and the sequence starts
on the fourth byte of the function.  Thus, this code contains a
gadget, 
having a starting address of {\tt 0x400f18},
that will copy the 64-bit value in register \raxreg{} to register \rdireg{}.

Your code for {\sc rtarget} contains a number of functions similar to
the \verb@setval_210@ function shown above in a region we refer to as
the {\em gadget farm}.  Your job will be to identify useful gadgets in
the gadget farm and use these to perform attacks similar to those you
did in Phases 2 and 3.

{\bf Important:} The gadget farm is demarcated by functions
\verb@start_farm@ and \verb@end_farm@ in your copy of {\tt rtarget}.
Do not attempt to construct gadgets from other portions of the program code.

\begin{figure}
A. Encodings of {\tt movq} instructions
\begin{center}
\small
\begin{tabular}{|c|c|c|c|c|c|c|c|c|}
\multicolumn{9}{l}{\texttt{movq $S$, $D$}} \\
\hline
Source & \multicolumn{8}{|c|}{Destination $D$} \\
\cline{2-9}
  $S$ & \verb@%rax@ & \verb@%rcx@ & \verb@%rdx@ & \verb@%rbx@ & \verb@%rsp@ & \verb@%rbp@ & \verb@%rsi@ & \verb@%rdi@ \\
\hline
\verb@%rax@ & \texttt{48 89 c0} & \texttt{48 89 c1} & \texttt{48 89 c2} & \texttt{48 89 c3} & \texttt{48 89 c4} & \texttt{48 89 c5} & \texttt{48 89 c6} & \texttt{48 89 c7} \\
\verb@%rcx@ & \texttt{48 89 c8} & \texttt{48 89 c9} & \texttt{48 89 ca} & \texttt{48 89 cb} & \texttt{48 89 cc} & \texttt{48 89 cd} & \texttt{48 89 ce} & \texttt{48 89 cf} \\
\verb@%rdx@ & \texttt{48 89 d0} & \texttt{48 89 d1} & \texttt{48 89 d2} & \texttt{48 89 d3} & \texttt{48 89 d4} & \texttt{48 89 d5} & \texttt{48 89 d6} & \texttt{48 89 d7} \\
\verb@%rbx@ & \texttt{48 89 d8} & \texttt{48 89 d9} & \texttt{48 89 da} & \texttt{48 89 db} & \texttt{48 89 dc} & \texttt{48 89 dd} & \texttt{48 89 de} & \texttt{48 89 df} \\
\verb@%rsp@ & \texttt{48 89 e0} & \texttt{48 89 e1} & \texttt{48 89 e2} & \texttt{48 89 e3} & \texttt{48 89 e4} & \texttt{48 89 e5} & \texttt{48 89 e6} & \texttt{48 89 e7} \\
\verb@%rbp@ & \texttt{48 89 e8} & \texttt{48 89 e9} & \texttt{48 89 ea} & \texttt{48 89 eb} & \texttt{48 89 ec} & \texttt{48 89 ed} & \texttt{48 89 ee} & \texttt{48 89 ef} \\
\verb@%rsi@ & \texttt{48 89 f0} & \texttt{48 89 f1} & \texttt{48 89 f2} & \texttt{48 89 f3} & \texttt{48 89 f4} & \texttt{48 89 f5} & \texttt{48 89 f6} & \texttt{48 89 f7} \\
\verb@%rdi@ & \texttt{48 89 f8} & \texttt{48 89 f9} & \texttt{48 89 fa} & \texttt{48 89 fb} & \texttt{48 89 fc} & \texttt{48 89 fd} & \texttt{48 89 fe} & \texttt{48 89 ff} \\
\hline
\end{tabular}

\end{center}
B. Encodings of {\tt popq} instructions
\begin{center}
\small
\begin{tabular}{|c|c|c|c|c|c|c|c|c|}
\hline
Operation & \multicolumn{8}{|c|}{Register $R$} \\
\cline{2-9}
 & \verb@%rax@ & \verb@%rcx@ & \verb@%rdx@ & \verb@%rbx@ & \verb@%rsp@ & \verb@%rbp@ & \verb@%rsi@ & \verb@%rdi@ \\
\hline
\texttt{popq $R$} & \texttt{58} & \texttt{59} & \texttt{5a} & \texttt{5b} & \texttt{5c} & \texttt{5d} & \texttt{5e} & \texttt{5f} \\
\hline
\end{tabular}

\end{center}
C. Encodings of {\tt movl} instructions
\begin{center}
\small
\begin{tabular}{|c|c|c|c|c|c|c|c|c|}
\multicolumn{9}{l}{\texttt{movl $S$, $D$}} \\
\hline
Source & \multicolumn{8}{|c|}{Destination $D$} \\
\cline{2-9}
  $S$ & \verb@%eax@ & \verb@%ecx@ & \verb@%edx@ & \verb@%ebx@ & \verb@%esp@ & \verb@%ebp@ & \verb@%esi@ & \verb@%edi@ \\
\hline
\verb@%eax@ & \texttt{89 c0} & \texttt{89 c1} & \texttt{89 c2} & \texttt{89 c3} & \texttt{89 c4} & \texttt{89 c5} & \texttt{89 c6} & \texttt{89 c7} \\
\verb@%ecx@ & \texttt{89 c8} & \texttt{89 c9} & \texttt{89 ca} & \texttt{89 cb} & \texttt{89 cc} & \texttt{89 cd} & \texttt{89 ce} & \texttt{89 cf} \\
\verb@%edx@ & \texttt{89 d0} & \texttt{89 d1} & \texttt{89 d2} & \texttt{89 d3} & \texttt{89 d4} & \texttt{89 d5} & \texttt{89 d6} & \texttt{89 d7} \\
\verb@%ebx@ & \texttt{89 d8} & \texttt{89 d9} & \texttt{89 da} & \texttt{89 db} & \texttt{89 dc} & \texttt{89 dd} & \texttt{89 de} & \texttt{89 df} \\
\verb@%esp@ & \texttt{89 e0} & \texttt{89 e1} & \texttt{89 e2} & \texttt{89 e3} & \texttt{89 e4} & \texttt{89 e5} & \texttt{89 e6} & \texttt{89 e7} \\
\verb@%ebp@ & \texttt{89 e8} & \texttt{89 e9} & \texttt{89 ea} & \texttt{89 eb} & \texttt{89 ec} & \texttt{89 ed} & \texttt{89 ee} & \texttt{89 ef} \\
\verb@%esi@ & \texttt{89 f0} & \texttt{89 f1} & \texttt{89 f2} & \texttt{89 f3} & \texttt{89 f4} & \texttt{89 f5} & \texttt{89 f6} & \texttt{89 f7} \\
\verb@%edi@ & \texttt{89 f8} & \texttt{89 f9} & \texttt{89 fa} & \texttt{89 fb} & \texttt{89 fc} & \texttt{89 fd} & \texttt{89 fe} & \texttt{89 ff} \\
\hline
\end{tabular}

\end{center}
D. Encodings of 2-byte functional nop instructions
\begin{center}
\small
\begin{tabular}{|ll|c|c|c|c|}
\hline
\multicolumn{2}{|c|}{Operation} & \multicolumn{4}{|c|}{Register $R$} \\
\cline{3-6}
 & & \verb@%al@ & \verb@%cl@ & \verb@%dl@ & \verb@%bl@ \\
\hline
 \texttt{andb} & \texttt{$R$, $R$} & \verb@20 c0@ & \verb@20 c9@ & \verb@20 d2@ & \verb@20 db@ \\
 \texttt{orb} & \texttt{$R$, $R$} & \verb@08 c0@ & \verb@08 c9@ & \verb@08 d2@ & \verb@08 db@ \\
 \texttt{cmpb} & \texttt{$R$, $R$} & \verb@38 c0@ & \verb@38 c9@ & \verb@38 d2@ & \verb@38 db@ \\
 \texttt{testb} & \texttt{$R$, $R$} & \verb@84 c0@ & \verb@84 c9@ & \verb@84 d2@ & \verb@84 db@ \\
\hline
\end{tabular}

\end{center}
\caption{Byte encodings of instructions.  All values are shown in hexadecimal.}
\label{fig:encode}
\end{figure}

\subsection{Level 2}

For Phase 4, you will repeat the attack of Phase 2, but do so on
program {\sc rtarget} using gadgets from your gadget farm.  You can
construct your solution using gadgets consisting of the following
instruction types, and using only the first eight x86-64 registers
(\raxreg{}--\rdireg{}).
\begin{description}
\item[{\tt movq}]: The codes for these are shown in Figure
  \ref{fig:encode}A. 
\item[{\tt popq}]: The codes for these are shown in Figure
  \ref{fig:encode}B. 
\item[{\tt ret}]: This instruction is encoded by the single byte {\tt 0xc3}.
\item[{\tt nop}]: This instruction (pronounced ``no op,'' which is
  short for ``no operation'')  is encoded by the single byte {\tt
  0x90}.  Its only effect is to cause the program counter to be
  incremented by 1.
\end{description}

{\bf Some Advice}:
\begin{itemize}
\item All the gadgets you need can be found in the region of the
  code for {\tt rtarget} demarcated by the functions \verb@start_farm@ and
  \verb@mid_farm@.
\item You can do this attack with just two gadgets.
\item When a gadget uses a {\tt popq} instruction, it will pop data
  from the stack.  As a result, your exploit string will contain a
  combination of gadget addresses and data.
\end{itemize}

\subsection{Level 3}

Before you take on the Phase 5, pause to consider what you have
accomplished so far.
In Phases 2 and 3,
you caused a program to execute machine code of your
own design.  If {\sc ctarget} had been a network server, you could
have injected your own code into a distant machine.  In Phase 4, you
circumvented two of the main devices modern systems use to thwart
buffer overflow attacks.  Although you did not inject your own code,
you were able inject a type of program that operates by stitching
together sequences of existing code.
You have also gotten 95/100 points for the lab.  That's a good score.
If you have other pressing obligations consider stopping right now.

Phase 5 requires you to do an ROP attack on {\sc rtarget} to invoke
function {\tt touch3} with a pointer to a string representation of
your cookie.  That may not seem significantly more difficult than using an
ROP attack to invoke {\tt touch2}, except that we have made it so.
Moreover, Phase 5 counts for only 5 points, which is not a true
measure of the effort it will require.  Think of it as more an extra
credit problem for those who want to go beyond the normal expectations
for the course. 

To solve Phase 5, you can use gadgets in the region of the code in {\tt rtarget}
demarcated by functions \verb@start_farm@ and
\verb@end_farm@.  In addition to the gadgets used in Phase 4,
this expanded farm includes the encodings of different {\tt movl} instructions, as
shown in Figure \ref{fig:encode}C.  The byte sequences in
this part of the farm also contain 
2-byte instructions that serve as {\em
  functional nops}, i.e., they do not change any register or memory values.
These include instructions, shown in Figure \ref{fig:encode}D, such as 
\verb@andb %al,%al@,
%
that operate on the low-order bytes of some of the registers but do
not change their values.

{\bf Some Advice:}
\begin{itemize}
\item You'll want to review the effect a {\tt movl} instruction has on
  the upper 4 bytes of a register, as is described on page 183 of the text.
\item The official solution requires eight gadgets (not all of which
  are unique).  
\end{itemize}


Good luck and have fun!

\appendix

\section{Using {\sc Hex2raw}}
\label{app:hex2raw}

{\sc Hex2raw} takes
as input a {\em hex-formatted} string.  In this format, each byte
value is represented by two hex digits.  For example, the string
``{\tt 012345}'' could be entered in hex format as ``{\tt 30 31 32 33
34 35 00}.'' (Recall that the ASCII code for decimal digit $x$ is {\tt
0x3}$x$, and that the end of a string is indicated by a null byte.)

The hex characters you pass to {\sc hex2raw} should be separated by
whitespace (blanks or newlines). We recommend separating different
parts of your exploit string with newlines while you're working on
it. {\sc hex2raw} supports C-style block comments, so you can
mark off sections of your exploit string. For example:

\begin{tty}
48 c7 c1 f0 11 40 00 /* mov    $0x40011f0,%rcx */
\end{tty}
%% $ (De-confuse text editor's autoformatting)

Be sure to leave space around both the starting and ending comment strings (``\verb@/*@'', ``\verb@*/@''),
so that the comments will be properly ignored.

If you generate a hex-formatted exploit string in the file {\tt
exploit.txt}, you can apply the raw string to {\sc ctarget} or {\sc
  rtarget} in several
different ways:
\begin{enumerate}
\item You can set up a series of pipes to pass the string through {\sc hex2raw}.
\begin{tty}
    unix>{\em cat exploit.txt | ./hex2raw | ./ctarget}
\end{tty}
\item
You can store the raw string in a file and use I/O redirection:
\begin{tty}
    unix>{\em ./hex2raw < exploit.txt > exploit-raw.txt}
    unix>{\em ./ctarget < exploit-raw.txt}
\end{tty}
This approach can also be used when running from within
{\sc gdb}:
\begin{tty}
    unix>{\em gdb ctarget}
    (gdb){\em run < exploit-raw.txt}
\end{tty}
\item
You can store the raw string in a file and provide the file name as a
command-line argument:
\begin{tty}
    unix>{\em ./hex2raw < exploit.txt > exploit-raw.txt}
    unix>{\em ./ctarget -i exploit-raw.txt}
\end{tty}
This approach also can be used when running from within {\sc gdb}.
\end{enumerate}


\section{Generating Byte Codes}
\label{app:bytecode}

Using {\sc gcc} as an assembler and {\sc objdump} as a disassembler
makes it convenient to generate the byte codes for instruction
sequences.  For example, suppose you write a file \verb@example.s@
containing the following assembly code:
\begin{ccode}
\begin{tty}
\input{example.s}
\end{tty}
\end{ccode}

The code can contain a mixture of instructions and data.  Anything to
the right of a `\verb@#@' character is a comment.  

You can now assemble and disassemble this file:
\begin{tty}
    unix>{\em gcc -c example.s}
    unix>{\em objdump -d example.o > example.d}
\end{tty}
The generated file {\tt example.d} contains the following:
\begin{ccode}
\begin{tty}
\input{example.d}
\end{tty}
\end{ccode}
The lines at the bottom show the machine code generated from the
assembly language instructions.  Each line has a hexadecimal number
on the left
indicating the instruction's starting address (starting with 0), while the hex digits
after the `\verb@:@' character indicate the byte codes for the
instruction.  Thus, we can see that the instruction {\tt push
\$0xABCDEF} has hex-formatted byte code {\tt 68 ef cd ab 00}.

From this file, you can get the byte sequence for the code:
\begin{ccode}
\begin{tty}
\input{example.txt}
\end{tty}
\end{ccode}
This string can then be passed through {\sc hex2raw} to generate an
input string for the target programs..  Alternatively, you can edit
example.d to omit extraneous values and to contain C-style comments for
readability, yielding:
\begin{ccode}
\begin{alltt}
\input{example.ed}
\end{alltt}
\end{ccode}
This is also a valid input you can pass through {\sc hex2raw} before
sending to one of the target programs.

\bibliography{refs}

\end{document}
